\week \ covers sections of chapter 14 in the textbook. Topics include:

\begin{itemize}
	\item First Law of Thermodynamics
	\item thermodynamic processes and cycles
	\item efficiency
	\item heat engines vs heat pumps
\end{itemize}

\begin{enumerate}
	\setlength\itemsep{2 in}
	
\item 
On a cold day, you rub your hands together to warm them up. If you press your hands together with a force of \SI{5}{\newton} and the coefficient of friction between your hands is $\mu = 0.45$ and your hands move an average distance of \SI{12}{cm}, then how much has the internal energy of your hands changed?

\item
Express the first law of thermodynamics as a statement about power and the rate of energy change instead of the total energy changes. 

\item When you exercise, you do work on something else. What does this mean about the sign of the work done on you? When you exercise, your body gives off heat to the surroundings (through evaporation mostly, but also some radiation and convection). what does this mean about the sign of the heat flow from you? If you do work at a rate of \SI{200}{\watt} and you radiate heat at a rate of \SI{800}{\watt}, the what is the rate of internal energy change of your body? If you exercise for \SI{30}{minutes}, then how much has your internal energy changed? How many bowls of cornflakes is this? 

\item 
In the notes I claim that $\text{Work} = \text{Pressure}\times\Delta\text{Volume}$. Show clearly how the units here work out to be equal.

\item
\SI{14}{\kilo\joule} of heat flow into a gas cylinder with a piston and the internal energy of the gas increases by \SI{42}{\kilo\joule}. How is this possible? Did the volume get bigger or smaller?

\item 
Rank the thermodynamic processes represented in the figure below in order of the work done from least to greatest. Rank positive work as higher than negative work.\\
\includegraphics{figures/PV-ranking.pdf}
\vspace{-2in}
\item
A monatomic ideal gas initially at room temperature (\SI{20}{\celsius}) undergoes an isobaric expansion from an initial volume of \SI{1}{\liter} to \SI{2}{L} at a pressure of \SI{2}{atm}. Then the gas goes through an isochoric process where the pressure lowers to \SI{1}{atm}. Draw a PV diagram of this process, and then calculate the total work done on the gas in these two processes. 

\item
A monatomic ideal gas initially at room temperature (\SI{20}{\celsius}) undergoes an isochoric process where the pressure changed from \SI{2}{atm} to \SI{1}{atm}. Then the gas goes through an isobaric expansion from an initial volume of \SI{1}{\liter} to \SI{2}{L}. Draw a PV diagram of this process, and then calculate the total work done on the gas in these two processes. 

\item
Sketch a PV diagram of a thermodynamic cycle that contains an two isothermal expansions and and two isochoric processes. First do this as a heat engine, then do it as a refrigerator. Specify for each step, when heat is going in or out, and when work is being done on the gas or on the surroundings.

\item
The following PV diagram is a wrong representation of the Otto Cycle, which is a model of a gasoline engine. What are the problems in the diagram below.\\
\includegraphics{figures/PV-otto-cycle-mistake.pdf}

\item
Rank the following cycles by the net work done by the system onto the surroundings from least to greatest. Rank positive work on the surroundings as higher than negative work (which would be done on the engine like in a heat pump). Which of these are heat engines and which are heat pumps?\\
\includegraphics{figures/PV-cycle-ranking.pdf}

\item
A heat engine operates with two isobaric processes and two isochoric processes. The pressure varies from \SI{1}{atm} and \SI{4}{atm} and the volume varies from \SI{0.200}{m^3} to \SI{0.800}{m^3}. Draw a PV diagram and specify the directions of these transitions. How much work is done by the engine in one cycle? What is the net heat flow into the engine per cycle? What is the heat in and the heat out? What is the efficiency of this engine?\giantskip

\item
What is the difference between a heat pump and an air conditioner? 

\item
The definition of efficiency from the notes is: \[e = \frac{W_{out}}{Q_{in}}\]

Also from the first law of thermo, we had: \[Q_{in} = W_{out}+Q_{out}\]

Combine these two equations to get an expression for efficiency only in terms of heat in and out (\emph{Hint: substitute for Work.})

\item
Follow up on the previous problem. Heat engines can operate at many different efficiencies depending on the cycles and how much heat goes in and out, but there is a cycle that represents an ideal heat engine that is the best efficiency possible. It is called the \emph{Carnot Cycle}. The details of this cycle are not important but the fundamental take away is that the heat that comes in ($Q_{in}$) comes from a source that is at temperature $T_H$, and the heat that goes out ($Q_{out}$) goes to a temperature of the surroundings at $T_C$. For a heat engine that operates between these temperatures (like the Carnot cycle) the ratio of the Heat out and in is equal to the ratio of the temperatures: \[\frac{Q_{out}}{Q_{in}}=\frac{T_C}{T_H}\]
This is a long lead up to the task of use this proportion to find an expression for the efficiency of an ideal heat engine in terms of the temperatures between which the engine operates. (\emph{Hint: Make a substitution in the equation from the answer to the previous problem.})

\item
In the engine of a car, suppose the combustion of gasoline and air that takes place produces temperatures that are around \SI{3000}{\celsius} while the exhaust gas exits the engine at a temperature of around \SI{1000}{\celsius}. Forget about the actual cycle that a gasoline engine undergoes, if this were an ideal process what is the efficiency that this engine can achieve. If you would improve the design to the extent that the exhaust gasses left the engine at room temperature and it still operated as an ideal heat engine, then what would the efficiency be? 


\newpage 

\ % The empty page

\newpage

\end{enumerate}