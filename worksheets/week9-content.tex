\week \ covers sections 1, 8, and 9 of chapter 19 in the textbook. Topics include:

\begin{itemize}
	\item magnetic field and everyday phenomena
	\item magnetic field from a line current
	\item magnetic field from a coil or dipole
	\item solenoids
	\item the right hand rule
\end{itemize}

\begin{enumerate}
\setlength\itemsep{3 in}

\item
The hardest part of describing magnetic fields is describing their direction in relation to the direction of other quantities. So the guide below will hopefully help us all use the same terminology so that we can use a consistent system to describe things. There are two common ways of describing the direction of things: the \emph{cardinal directions} (North, South, East, West, Up, Down) and relative directions (Away, Toward, Right, Left, Up Down). Both of these are displayed in the figures below so take some time to see how these two perspectives, and two descriptions work.

\includegraphics[scale=0.8]{figures/magntic-field-directions.png}

\item
Draw a picture of a straight line current going to the right. What is the direction of the magnetic field south of the wire? What is the direction of the magnetic field north of the current? What is the direction of the magnetic field directly above the current (out of the page)? What is the direction of the magnetic field directly below the current (into the page)? Now repeat this but with a current that is flowing south (toward you on the page)? If you put a compass in these locations, in what direction would the compass point? \hugeskip

\item
Draw a plot of the strength of the magnetic field at a function of distance away from a  long straight current. First, what current would have to be in the wire so that the magnetic field strength was \SI{1}{\tesla} at a distance \SI{1}{cm} away from the wire? Plot this in \SI{0.5}{cm} increments.

\item
How far away from the wire in the above problem do you have to be to have a magnetic field strength of 1\% of the strength at \SI{1}{cm} away? Work this as a proportion problem.

\item
Draw two parallel wires that have current traveling in the same direction (draw them both going upward). The wires are separated by a distance of \SI{1}{cm} and they each carry a current of \SI{10}{\ampere}. Find the magnitude and direction of the magnetic field \SI{0.5}{cm} to the left of the wire on the left and midway between the wires, and \SI{0.5}{cm} to the right of the wire on the right. What is the magnetic field produced by the wire on the left at the location of the wire on the right?

\item
Draw the same problem as above, but with the currents traveling in opposite directions. Make the current on the left go up, and the current on the right go down.

\item
Now draw a \SI{10}{\ampere} current loop and have the current go counter clockwise. What is the direction of the magnetic field at the center of the loop? Use both versions of the right hand rule to show yourself this. In what direction does the magnetic field point in the region outside of the loop. Draw the current with some perspective so that the magnetic field points to the right. What radius of loop is necessary to produce \SI{1}{\tesla} at the center? If the radius of the loop was \SI{20}{cm} then how many loops would you have to have to make \SI{1}{\tesla} of magnetic field at the center?

\item
Now draw a loop of current, but draw it so that the magnetic field point northward. On the east side of this loop, in what direction is the current going? What about the west side? Now draw in magnetic field lines, two loops for each side of this current and draw an axis through the center of the loop. Draw in three different locations how a compass would orient itself in this magnetic field. This pattern is known as a dipole and we will need this again next week to hold onto this drawing.

\item
Two separate current loops are positioned so that they form concentric circles. The radius of the smaller loop is \SI{1}{cm} and the radius of the larger is \SI{1.5}{cm}. The larger loop has a current of \SI{10}{\ampere}. If the magnetic field at the center of the loops is \SI{0}{\tesla}, then what is the magnitude and direction of the current in the inner loop.







\newpage 

\ % The empty page

\newpage

\end{enumerate}