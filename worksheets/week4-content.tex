\week \ covers sections of chapter 16 in the textbook. Topics include:

\begin{itemize}
	\item charge
	\item electric field and electric force
	\item motion of particles within a constant electric field
\end{itemize}

\begin{enumerate}
\setlength\itemsep{2 in}
	
\item Let's work on units.
\begin{itemize}
	\setlength\itemsep{1 in}
    \item What are the units of electric field? Use the equation $\vec{F}=q\cdot \vec{E}$
    \item What are the units of the Coulomb constant, $k$? Use the equation $E=\dfrac{k q_0}{r^2}$ or $F=\dfrac{k q_o q_1 }{r^2}$.
    \item What are the units of the vacuum permittivity constant, $\epsilon_0$? Use the equation $k=\dfrac{1}{4 \pi \epsilon_0}$.
\end{itemize}

\item
An elementary charge is the charge of either an electron or proton, and interestingly they are the same amount but different sign. How many elementary charges are in one Coulomb of charge? How many Coulombs of charge is one elementary charge? How many electrons would there be in \SI{1}{\micro\coulomb} of charge? \SI{1}{\nano\coulomb}?

\item 
What is the electric field of a \SI{10}{\micro\coulomb} point charge \SI{1}{\meter} away from the point charge? 

\item
Where is the electric field of a \SI{10}{\micro\coulomb} point charge \SI{50}{\newton/\coulomb}? 

\item 
How much charge is in a point charge if the electric field is \SI{1}{\newton/\coulomb} \SI{1}{\meter} away?

\item
If you double the distance away from a point charge, by what factor do you change the electric field strength? Triple? Half? Quarter? 

\item
If the electric field doubles, by what factor do the distances between these locations differ? What if the electric field is 10 times smaller?

\item
A \SI{3}{\micro\coulomb} and a \SI{5}{\micro\coulomb} are \SI{5}{\meter} away from each other. What is the magnitude of the electric field produced by the \SI{3}{\micro\coulomb} charge \SI{5}{\meter} away from it? What is the force on the \SI{5}{\micro\coulomb} charge? If the \SI{5}{\micro\coulomb} charge was instead a \SI{9}{\micro\coulomb} charge, what is the force on it? In what direction does this force point?

\item
Lets see if Newton's Third Law applies to the electric force. In the last problem you found the electric field of the \SI{3}{\micro\coulomb} charge and then multiplied by the other charge. But this time find the electric field of the \SI{5}{\micro\coulomb} charge \SI{5}{\meter} away? Is it the same? What is the force on the \SI{3}{\micro\coulomb} charge? Is it the same? What is its direction? Is this consistent with Newton's Third Law?

\item
A \SI{+20}{\micro\coulomb} charge is \SI{2}{\meter} away from a \SI{-5}{\micro\coulomb} charge. What is the electric field halfway between the two charges? What is it \SI{1}{\meter} away from the \SI{+20}{\micro\coulomb} charge on the side opposite the \SI{-5}{\micro\coulomb} charge. What is the electric field \SI{1}{\meter} away from the \SI{-5}{\micro\coulomb} charge on the opposite side as the \SI{+20}{\micro\coulomb} charge? Draw a picture of all of this to keep the locations straight. \giantskip

\item
For the charge distribution above, can you find where the electric field would equal zero? Think first about which of the three possibilities along the axis where these charges exist. Then set up an equation and solve for the distance away from the negative charge where the field is zero.

\item
I have collected \SI{0.001}{\coulomb} of charge on a sheet that has an area of \SI{100}{\centi\meter\squared}. What is the electric field \SI{1}{mm} away from this sheet? What is it \SI{1}{cm} away? Is is the same value \SI{1}{m} away? (\emph{Thing about the inside out versions of this problem too.})

\item
When I put charge on a large sheet, it is common to refer to the \emph{charge density} of the sheet rather than the total charge and the total area. The charge density, $\sigma$ is defined like this: \[\sigma = \frac{Q}{A}.\] Substitute this into the equation for the electric field of a large charged sheet.

\item
You have two large circular sheets of diameter \SI{15}{cm}. You orient the sheets so they are parallel to one another. Draw a picture of this. Now imagine moving electrons from on plate and placing them on the other plate. If you do this with a billion electrons then how much charge will be on each plate? This kind of device is called a \emph{capacitor} and we will talk more about this next week. What will be the electric field between the plates? What will be the electric field outside of the positive plate? What will be the electric field outside of the negative plate? \bigskip

\item
A capacitor has a charge density of \SI{1e-6}{\coulomb/m^2}. If the positively charged plate is on the left, and the negatively charged plate is on the right and they are separated by a distance of \SI{1}{cm}. You place a proton in the center between the plates. What electric force does this charge experience? In what direction does it experience the force? What about if this were an electron? 

\item
Look up the mass of a proton and an electron (book, google, etc). If you were to let go of either the proton or electron in the previous problem then what would the acceleration of that particle be? Write a general expression for the acceleration of a charged point particle if the electric force is the only force acting on it. 

\item
You have a capacitor where a total charge of \SI{10}{\micro\coulomb} has moved between the plates, and the plates have an area of \SI{0.01}{m^2}. The plates are separated by a distance of \SI{1}{cm}. If a proton is released from rest from the positive plate, then how fast is it going when it gets to the negative plate?

\newpage

EXTRAS!

\item
A positive and a negative charge will clearly attract on another until they collide, but if you put them perfectly lined up in an electric field, then they will maintain some distance from each other. Draw a uniform electric field that points to the right. Place a negative and positive charge within this field in an arrangement so that the charges will not move (and the force on them is zero). What distance apart are they. Assume it is a proton and an electron. 


\newpage 

\ % The empty page

\newpage

\end{enumerate}