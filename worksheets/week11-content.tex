\week \ covers sections of chapter 22 in the textbook. Topics include:

\begin{itemize}
	\item electromagnetic waves
	\item index of refraction
	\item doppler shift

\end{itemize}

\begin{enumerate}
\setlength\itemsep{2 in}

\item
What is the range of wavelengths of human vision? What is the range of frequencies? Approximately what wavelength is red? Green? Violet? A wave that has a wavelength of \SI{1}{nm} could be used for what purpose? What is its frequency? Wifi usually works around a frequency of \SI{2.4}{\giga\hertz} which is the same as a consumer microwave oven. What wavelength is this? WBHM is the Public Radio station in Birmingham and you can find it on the radio dial at \SI{90.3}{\mega\hertz}. What wavelength is this? \giantskip

\item
What frequency has the same magnitude as its wavelength? What region is this in?

\item
The sun is \SI{93000000}{miles} away from earth. If the sun suddenly burned out, how long would it take for us to know?

\item
One light-year is the distance that light travels in a year. How many meters is this?

\item
What is the speed of light in water where $n=1.33$? What about the speed of light in diamond where $n=2.42$? 

\item
The speed of light in some unknown material is measured to be \SI{1.3e6}{m/s}. What is the index of refraction?

\item
If the frequency of light in water is \SI{1e14}{\hertz} in water then what is it in diamond? What is the wavelength in water? What is the wavelength in diamond?

\item
Show that the quantity and units work out for the relationship between the speed of light in vacuum and the permittivity and permeability of free space: \[c = \frac{1}{\sqrt{\epsilon_0\cdot\mu_0}}\]

\item
The simplest model for an electromagnetic wave to take is a sine wave. The general form for using the sine wave and having the parameters match those of wavelength and period that we have discussed can be written like this:
\[E(x, t) = E_0 \sin\left(\frac{2 \pi}{\lambda}x - \frac{2\pi}{T}t\right)\] 
This is just a function that gives you the value of the electric field at a particular place ($x$) at a particular time ($t$). For an electromagnetic wave of wavelength \SI{100}{\nano\meter}, what is the period? Sketch two wavelengths of this wave on a graph when $t=0$. Then sketch another plot 1/3 of a period later.


\item
\emph{Doppler Effect.} Imagine yourself approaching a source of light and think of the light as waves. You are traveling in the opposite direction as the light propagation. You can see from some of the above graphs of waves traveling that if you are doing this, then the period of time between when crests of the light are detected by you will be shorter than for a stationary observer. This means the frequency you observe will be higher than the frequency of the source. The faster you go the higher this frequency shift will be. The opposite effect will happen if you are traveling \emph{away} from the source of light; you will observe a lower frequency. The equation relating the frequency you observe, $f_o$, the frequency of the source $f_s$ and the \emph{relative velocity} between source and observer $v_{rel}$ is 
\[f_o = f_s\sqrt{\frac{1+\frac{v_{rel}}{c}}{1-\frac{v_{rel}}{c}}}\]

In this equation $v_{rel}$ is negative when the source and observer are moving away from each other and positive when they are approaching each other.

Using this equation, calculate how fast you would have to be going in your car in order for a red light to appear green.\giantskip

\item
The speed of light is very very fast, and many times the relative velocities between source and observer are much smaller than that. The equation above can be simplified in cases when $v_{rel}<<c$ to a much easier equation: 
\[f_s = f_o\left(1+\frac{v_{rel}}{c}\right)\]
The radar of a police officer's radar gun emits microwave radiation at about \SI{3e10}{\hertz}. The officer is at driving at \SI{35}{mph} some very reckless person is driving toward him at \SI{55}{mph} in a \SI{35}{mph} zone. 
\begin{enumerate}
	\setlength\itemsep{2 in}
	\item What is the relative velocity between the officer and the driver?
	\item What is the frequency of the radar that the speeding car observes?
	\item When this radiation hits the speeding car, it reflects back to the officer, but the reflected light is essentially re-emitted from the speeding car at the doppler shifted frequency it ``observed''. So now the speeding car is emitting radar (it is a now source) and the officer is observing this radiation, but since they are moving toward each other, it is doppler shifted again. What frequency of radar does the officer observe now?
	\item Actually measuring this frequency is hard, since they are so close together, but what can be done is measuring the \emph{beat frequency} of the officer's emitted radar, and the observed beam reflected from the speeder. What is the beat frequency of these two waves? 
\end{enumerate} 

\item
At what relative speed does the approximate form of the doppler shift equation give an error of 1\%. You should make a table (or even better an Excel sheet) and try out several values for $v_{rel}$ to find out. Choose any $f_s$ or just use the ratio of $f_o/f_s$.

\newpage 

\ % The empty page

\newpage

\end{enumerate}