Week 1 covers sections 1-5 of chapter 13 in the textbook. Topics include
\begin{itemize}
	\item temperature and measurement scales
	\item measurements of amount and density
	\item the ideal gas law
	\item kinetic theory of gas 
\end{itemize}


\begin{enumerate}
\setlength\itemsep{2 in}

\item
The Celsius temperature scale is based on the \emph{triple point} of water, but it is more common to think of it as being \qty{0}{\celsius} when water freezes and \qty{100}{\celsius} when water boils at \qty{1}{atm} of pressure. But the Fahrenheit scale is more well known to us so lets do some conversion of common Fahrenheit temperatures. \qty{105}{\fahrenheit}, \qty{98.6}{\fahrenheit}, \qty{72}{\fahrenheit}, \qty{32}{\fahrenheit}, \qty{0}{\fahrenheit}. Keep going down in Fahrenheit, and see if you can find a Fahrenheit temperature that gives you  the same number in Celsius. Make sure you can go backwards and convert some Celsius temperatures back to Fahrenheit.


\item
If I only tell you a \emph{change} in Fahrenheit temperature of a substance but not the actual temperature, then you can figure out the corresponding change in Celsius, but still not the actual temp. A change in temperature measured in Fahrenheit is 1.8 times bigger than the change measured in Celsius. So if the temperature increased by \SI{30}{\fahrenheit}, then by how much does the temperature change in Celsius? What does this mean about the "size" of a Celsius degree vs. the "size" of a Fahrenheit degree? Which one represents a larger change in temperature?

\item
The kelvin temperature scale is designed as an \emph{absolute} temperature scale, meaning the lowest temperature any object could theoretically be is set to \SI{0}{\kelvin}. The size of a Kelvin degree is the same as the size of a Celsius degree, so that a \SI{20}{\celsius} change in temperature is the same as a \SI{20}{\kelvin} temperature change. Absolute zero in the Kelvin Scale is set to \SI{-273.15}{\celsius}. So, what is \SI{0}{\celsius} in Kelvin? What is \SI{20}{\celsius} in Kelvin. What is \SI{70}{\kelvin} in Celsius? What is normal human body temperature in K?

\item
What is absolute zero in the Fahrenheit temperature scale? Find this by using $T_C = -273.15$ first if you want, but then try using a substitution for $T_C$ that will give you an expression for finding any Fahrenheit temperature given a Kelvin one.

\item
What is the molecular weight of Carbon-12? Find a periodic table to help. How many protons are in Carbon-12? How many neutrons? What about the number of protons in Carbon-14? What about the number of neutrons in Carbon-14?

\item 
How many atoms are in a mole of Helium? How many atoms are in a mole of Carbon-12? What is the mass of a mole of Helium? What is the mass of a mole of Carbon-12? 

\item
What is the mass of a single CO$_2$ molecule? What is the mass of a mole of CO$_2$?

\item
What is the mass of a mole of dry air which is 78\% N$_2$, 21\% O$_2$, and 1\% Ar?

\item
A balloon is filled with \SI{0.4}{\mole} of helium so that its volume is \SI{0.010}{\cubic \meter}. 
\begin{itemize}
	\setlength\itemsep{1 in}
	\item Find the number of atoms.
	\item Find the number density.
	\item Find the mass density.
	\item Estimate the average distance between atoms. To do this, fine the \emph{volume per particle}, and then treat that volume like a cube and find the side length of the cube. Draw a picture of this model and use that to justify your approximation.
\end{itemize}

\item
You have a pound of feathers and a pound of lead.
\begin{itemize}
	\item Which one weighs more?
	\item Which one has more mass?
	\item Which one has the greater volume?
	\item Which one contains a larger number of moles?
	\item Which one contains a larger number of atoms?
	\item Which one contains a larger number of protons and neutrons?
\end{itemize}

\item 
You check your car tire pressure and see that the pressure is \SI{25}{lb/in^2}. What is this in Pascal? (You'll need to look up a conversion factor). This is a gauge pressure, so what is the absolute pressure in the tire?

\item 
You check you car tire pressure when it is \SI{15}{\celsius} and it is \SI{25}{lb/in^2}. By what factor do you increase the number of particles in the tire so that the pressure becomes that \SI{30}{lb/in^2}? (\emph{Hint: The volume and temperature do not change.})

\item
The gas pressure inside of a 1 liter sealed container at room temperature is \SI{1}{atm}. How many molecules are inside? How many moles of molecules?

\item
If the pressure inside a tank is \SI{1}{atm} when the temperature is \SI{100}{\kelvin}, then what is the pressure when the temperature rises to \SI{200}{\kelvin}? 

\item
If the pressure inside a tank is \SI{1}{atm} when the temperature is \SI{100}{\celsius}, then what is the pressure when the temperature rises to \SI{200}{\celsius}? \emph{CAREFUL!}

\item
A gas is in a sealed container. By what factor does the pressure change if 
\begin{itemize}
	\setlength\itemsep{1 in}
	\item the volume is doubled?
	\item the temperature is tripled?
	\item the volume is double and the temperature is tripled?
	\item the volume is halved?
\end{itemize}

\item
You are standing in a room at atmospheric pressure and room temperature. You estimate the room to be \SI{10}{\meter} wide by \SI{15}{\meter} long by \SI{2}{\meter} high. How many moles of gas are in the room?



%\item
%number density to mass density relationship?




\end{enumerate}