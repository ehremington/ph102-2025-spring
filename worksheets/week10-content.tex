\week \ covers sections of chapter 19 in the textbook. Topics include:

\begin{itemize}
	\item magnetic force on a moving charge
	\item magnetic force on a loop
	\item circular path of a charge in a uniform magnetic field
	\item mass spectrometers and cyclotrons
	\item torque on a dipole and electric motors
\end{itemize}

\begin{enumerate}
\setlength\itemsep{3 in}

\item
Draw several examples of a magnetic field pointing in a certain direction. Populate each example with several charges moving in a certain direction and for each one find the direction of the magnetic force on that particle. Make sure you put in both positive and negative charges. Put in some cases where the magnetic force is zero. 

\clearpage

\item
Now think backwards. If the force on a particle moving to the right is upward, then what direction is the magnetic field it is within? What about a particle moving up with a force to the left? If a particle in in a magnetic field that points down, and it is experiencing a force south, then in which direction is its velocity? Think of some more variations on this.

\item
If an electron is experiencing a magnetic force to the right from a field pointed north, what direction is it moving? Think of some other combinations of this problem?

\item
How fast would a proton have to be traveling to experience a \SI{1}{\newton} force in a magnetic field of \SI{1}{\tesla}? How fast would \SI{1}{\coulomb} of charge need to travel to experience the same?

\item
In Birmingham, Earth's magnetic field has a magnitude of \SI{50}{\micro\tesla} and is directed \ang{70} below the horizontal and pointing north. Find the magnetic force on an oxygen ion moving east at 250 m/s. Compare the magnitude of the magnetic force with the ion's weight, \SI{5.2e-25}{\newton}, and to the electric force on it due to the Earth's fair weather electric field (\SI{150}{N/C} downward).

\item
How much current would be necessary to exert a \SI{1}{\newton} force on \SI{5}{cm} of wire within a magnetic field of \SI{0.5}{\milli\tesla} if the current was directed at a \ang{30} angle to the magnetic field? Choose a direction of the magnetic field and draw a picture of this, indicating the direction of the current and the direction of the force on the current. If the current were rotated, what is the maximum force that could be exerted on it?

\item
If a \SI{100}{\meter} long power line is carrying \SI{150}{\ampere} of current south in Birmingham (see the description above), then what is the magnetic force on this line? Since current is really electrons flowing in the opposite direction as the ``current'' does this change the direction of the force on the line?

\item
Charged particles in magnetic fields tend to travel in circular paths. A \SI{1}{\micro\coulomb} particle with a \SI{0.01}{\micro\gram} mass is traveling with a speed of \SI{1}{km/s} in a magnetic field. The particle travels in a circular path that has a radius of \SI{0.5}{\meter}
\begin{enumerate}
	\setlength\itemsep{2 in}
	\item What is its radial acceleration?
	\item What radial force is needed to cause this particle to travel in this particular path at this speed?
	\item If the magnetic force is the only force acting on this particle, then how large is that force?
	\item What magnetic field is necessary to provide this large of a magnetic force?
	\item If the particle is making counterclockwise loops within the magnetic field when viewed from above then in which direction does the magnetic field point?
	\item How long does this particle take to complete one full circle of its path around? (This is known as the period of motion.)
	\item How many times does the particle complete a full circle in one second? (This is known as the frequency.)
\end{enumerate}


\item
A mass spectrometer is a device for accelerating ions with parallel plates of different voltage and sending them into a magnetic field. The ions travel through a velocity selector so that they all have the same speed entering the magnetic field. Draw a picture of this with the ions accelerated to the right between plates and then into a magnetic field that is into the page. What will the path of the particles look like when they enter the magnetic field? Suppose we have two types of particles with different masses like $^6$Li$^+$ and $^7$Li$^+$ which are isotopes (same number of protons but different number of neutrons). For these particles going through the same magnetic field, what is the ratio of the radii of these two particles? If the radius of orbit of $^6$Li$^+$ is \SI{8.4}{cm} what is the radius of orbit of $^7$Li$^+$? If the speed of the particles entering the magnetic field is \SI{1e6}{m/s}, then what is the magnitude of the magnetic field?

\item
Two wires are separated by \SI{1}{cm} and carry \SI{100}{\ampere} each, but in opposite directions. Draw these going North/South with the North current on the left, just so we all have the same picture. What is the magnetic field created by the northward current at the location of the southward current (magnetude and direction)? What is the force on the southward current due to the northward currents magnetic field? What about the force of the southward current on the northward current? Does Newton's Third Law apply here?

\item
What about currents that travel in the same direction? Do they attract or repel one another and why?

\item
A \SI{100}{\ampere} current is flowing northward. A \SI{0.5}{\micro\coulomb} charge has a velocity of \SI{100}{m/s} northward. What is the magnetic force on the particle? What if the particle is traveling south? What if the particle's charge is negative? 




\item
If you threw a conducting metal rod to the left through a uniform magnetic field that also pointed left, so that the rod was perpendicular to the magnetic field as it flew through it, then what would happen to the negative charges in the rod that are free to move around? 

\item
An electric field and a magnetic field are arranged in space so that they cross each other at a right angle. Draw this so with the electric field pointing north and the magnetic field pointing east. A positive proton is moving into the page. Sketch the directions of the electric force on the particle and the magnetic force on the particle. The electric field is \SI{100}{N/C} and the magnetic field is \SI{50}{\milli\tesla}. How fast would the particle have to travel for the magnetic force and electric force to be equal to each other? This device is known as a velocity selector since only the particles with a specific velocity will pass through the center of a hole and out of the device. What will happen to particles that are traveling too quickly or too slowly?




\newpage 

\ % The empty page

\newpage

\end{enumerate}